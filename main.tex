\documentclass[a4paper,14pt]{article}
\usepackage{times}
\usepackage{cmap}     
\usepackage{mathtext}
\usepackage[T2A]{fontenc}
\usepackage[utf8]{inputenc}
\usepackage[english,russian]{babel}
\usepackage{indentfirst}
\usepackage{tempora}
\usepackage[14pt]{extsizes}
\frenchspacing

\author{Андреев Иван Васильевич}
\title{Рынок углеродных единиц.}
\date{\today}

\begin{document}

\maketitle

\section{Обзор литературы}
    \subsection{«К вопросу о правовом регулировании рынка углеродных единиц: сравнительно-правовой аспект» Хакимова Ж. А., Васильева А. А.}
    Данная статья посвящена правовой неопределенности углеродных единиц. Действительно, в наше время правительства развитых государств пытаются реализовать рынок углеродных единиц, ведь в теории он может крайне положительно повлиять на экологическую повестку в стране. Однако на практике возникает множество сложностей, связанных с культурными особенностями населения, внутреннем устройстве экономики и законодательства. Таким образом, для эффективного функционирования данного рынка требуется грамотное правовое регулирование, индивидуальное для каждого государства. Авторы статьи показывают решения данной проблемы, принятые различными странами. Одним из принятых на 2023 год международных документов является Рамочная конвенция, которая представляет квоту, как возможность предприятий совершать выбросы в атмосферу, идентичные одной тонне углекислого газа. Такого принципа, например, придерживается Бельгия и Германия. А в Великобритании, где активно применяются судебные прецеденты, в одном из судебных процессов углеродная единица была признана нематериальным имуществом. Авторы сообщают, что еще одним документом по данному вопросу является Киотский Протокол к UNCCCN, участником которого является Франция. Протокол представляет углеродную единицу материальным имуществом. Она может храниться на счете владельца и быть переведенной другой компании на счет, при заключении сделки. Авторы статьи рассматривают множество примеров реализации правового регулирования рынка углеродных единиц разными странами: США, Мексика, Южная Корея и другие. Данная статья является крайне полезной для написания исследования, посвященного рынку углеродных единиц. Она помогает читателю разобраться, что из себя законодательно представляет углеродная единица на реальных примерах. Однако, как мне кажется, информация, содержащаяся в статье, является неполной. Авторы не углубляются в особенности реализации «углеродного» права для каждого государства, а лишь обозначают форму единицы и ее оборота.
    \subsection{«Риски использования углеродной единицы как инструмента финансового рынка» Лукашенко И. В.}
    Рынок углеродных единиц находится лишь на стадии своего зарождения, поэтому крайне важно понимать риски его использования. Для этого обратимся к данной статье. В начале своей работы автор обозначает существующие системы торговли выбросами и нормативно-правовые акты различных государств. Об этом мы уже узнали из предыдущей статьи Хакимовой и Васильевой. Сравнивая две работы, хочется отметить, что у Лукашенко информация по этому поводу так же является неполной, однако она более структурированная, что является большим преимуществом. Далее автор переходит к описанию рисков, разделив их на три группы: экологические, инвестиционные и операционные. Под экологическими рисками автор подразумевает последствия негативного влияния человечества на природу. Инвестиционные риски представляют собой возможность потери дохода из-за вложений предприятий в экологически нейтральный капитал. Автор поясняет, что для стимуляции производителей обновлять технологии была введена система покупки и продажи углеродных единиц, крупными покупателями которых являются Мировой банк, Европейский банк реконструкции и развития с двумя углеродными фондами, NEFCO — Балтийский углеродный фонд, а также целые государства. Под операционными рисками автор подразумевает хакерские атаки, уклонение от налогов, мошенничество с выделяемыми квотами и другие. Содержание данной статьи так же является полезной для написания нашего исследования. Читатель погружается в проблемы реализации рынка углеродных единиц. Однако в своей работе автор не пишет о возможности нестабильности политической ситуации в стране, что является серьезным допущением, ведь политика оказывает крайне сильное влияние на экономику страны.
    \subsection{«Достижение углеродной нейтральности путем внедрения инструментов налогообложения» Александрова Ж.П., Кат С.А.}
    Авторы статьи рассказывают о еще одном механизме углеродного регулирования – трансграничном, который входит в пакет климатического законодательства, предложенного Европейской комиссией. Данный механизм предотвращает жульничество со стороны предприятий, перемещающих свои производства в страны со слабой экологической политикой. При продаже товаров в Европе, компании-импортеры обязаны покупать сертификат по цене за свои выбросы при производстве продукции на территории Европы. Авторы рассматривают минусы введения данного механизма, а также его влияние на российские компании. Например, в некоторых отраслях углеродный налоговой сбор может составить половину стоимости товара, что, безусловно, серьезно ухудшит доходы компаний. Посыл авторов работы заключается в том, что предлагаемые механизмы достижения углеродной нейтральности зачастую несут за собой серьезную угрозу для предприятий. Недостатком данной статьи я могу назвать отсутствие описания положительных последствий введения трансграничного механизма углеродного регулирования. Автор обращает внимания читателя лишь на его недостатки, хотя он действительно позволяет искоренить данный вид махинаций с углеродными единицами. Несмотря на это, статья так же может быть полезной для исследования.
    \subsection{«Углеродные единицы как объект бухгалтерского учета: признание, оценка, методики» Головач О. В.}
    В связи с особенностями и проблемами правового регулирования углеродных единиц возникают также и особенности их бухгалтерского учета. Данная статья поможет читателю разобраться в этом аспекте на примере законодательства Российской Федерации. Автор статьи проводит анализ возможности включения нового вида активов «углеродные единицы»
в состав традиционных для российской практики учета видов активов: основные средства, нематериальные активы и другие. Оказывается, углеродные единицы не могут быть отнесены ни к одному из существующих на данный момент активов. Это означает, что углеродные единицы требуют уникальной идентификации в правовом поле. Также автором были предложены варианты оценки углеродных единиц по справедливой стоимости. Данная статья позволяет исследователю снова погрузиться в проблемы правового регулирования рынка углеродных единиц. В частности, в проблемы ведения бухгалтерского учета компаниями. Однако автор не рассматривает возможность уклонения фирм от уплаты штрафов путем предоставления неверной информации о количестве имеющихся углеродных единиц.
\subsection{«Моделирование динамики цен единиц сокращенных выбросов на углеродном рынке» В. О. Тайлаков.}
В данной статье автор строит математическую модель прогнозирования цен на ЕСВ. Она может быть полезной для исследователя, так как дает понимание читателю, от чего зависит и как складывается цена на углеродные единицы. Однако стоит отметить, что автор использует сложные математические конструкции, например, линейные операторы, что сильно сужает целевую аудиторию автора. Также ученый в своей модели допускает, что скачки цен на углеродные единицы не зависят от курса европейских биржевых индексов.
\subsection{«Мировой углеродный рынок в стадии зарождения» Иванов Н. А.}
Данная статья в большей степени посвящена ценообразованию на рынке углеродных единиц. Автор поясняет, как складывается и от чего зависит цена углеродных единиц, обозначает виды систем ценообразований: система квот и торговли, система базовых показателей и кредитов. Также автор расматривает ценообразование на разных уровнях: международном – проекты совместного осуществления и механизм чистого развития, международные протоколы и документы, национальном, региональном. Во второй половине статьи автор приводит примеры организации рынка углеродных единиц различных государств и регионов – Китай, Калифорния, Европа. Данная статья является крайне полезной для исследователя. Сравнивая статьи Иванова и Тайлакова, можно прийти к выводу, что Иванов преподносит информацию более доступным способом и сильнее погружает читателя в особенности ценообразования углеродных единиц. Примеры реализаций углеродных рынков так же являются самыми подробными и глубокими именно в этой статье, среди всех рассмотренных.
\subsection{«Концепция углеродного регулирования и биржевого рынка углеродных единиц в регионах России» Небесная А. Ю.}
В данной статье А.Ю. Небесная описывает концепцию углеродного урегулирования и биржевого рынка углеродных единиц в регионах России. Автор отмечает, что глобальный углеродный рынок начал формироваться в 2005 году, когда вступил в силу Киотский протокол, определивший основные механизмы, которые направлены на взаимовыгодное сотрудничество стран-участниц. Целью данного взаимодействия является сокращение выбросов. Данный проект, согласно статье, позже трансформировался в “Схему зеленых инвестиций”, суть которой заключается в обязательных инвестициях средств в проекты, направленные на уменьшение негативного воздействия на окружающую среду. Данная схема находит своё отражение в международной системе ограничений на выбросы парниковых газов Cap-and-trade, которая очень подробно описывается А.Ю. Небесной, но слабо освещена другими авторами в выделенной мной академической литературе. Тем не менее, стоит отметить, что данная статья имеет недостатки. Автор лишь частично затрагивает концепцию углеродного регулирования именно в регионах России.
\subsection{«Формирование и развития российского рынка углеродных единиц» Скворцова М. А., Тяглов С. Г.}
В данной статье авторы рассматривают особенности реализации углеродного рынка в Российской Федерации. В начале статьи уделяется внимание предпосылкам формирования рынка углеродных единиц в России. Основной предпосылкой авторы называют введение Европейской комиссией трансграничного механизма регулирования углеродного рынка, о котором было сказано в статье Александровой и Кат. Российская Федерация должна вводить более жесткие внутренние ограничения на выброс парниковых газов, что компании-экспортеры могли ввозить товары в Европу без ограничений. Далее авторы обозначают тип реализации углеродного рынка в России – обязательный рынок с жестко регламентированным порядком функционирования. Авторы считают, что Российская Федерация сможет выполнить свои обязательства по достижению углеродной нейтральности экономики до 2060 года, что делает рынок углеродных единиц перспективным. Данная статья является полезной для исследования, так как предоставляет более глубокую информацию об особенностях формирования рынка в РФ, нежели статья Небесной. 
\subsection{«Особенности торговли углеродными единицами на финансовом рынке» Рубцов Б. Б., Лукашенко И. В. }
В данной статье Б.Б. Рубцов и И.В. Лукашенко выделяют особенности торговли углеродными единицами на финансовом рынке. Во-первых, авторы приводят разновидности углеродных единиц (Квота на выброс, Углеродный взаимозачет или кредит) согласно механизмам их сокращения, а также отмечают, что сокращение углеродных выбросов можно рассматривать как один из путей достижения экономичного и продуктивного хозяйствования. Более того, Б.Б. Рубцов и И.В. Лукашенко уделяют внимание тому, что в данной работе не рассматриваются отрицательные стороны инвентаризации выбросов, что является допущением. Далее в работе приводится обзор существующих рынков и объединений. Отмечается, что в мире существует лишь одна международная – европейская, работающая c 2005 года, обязательная система торговли выбросами – European Union Emissions Trading Systems. Авторы детально и структурировано подходят к описанию данной организации и её деятельности, разбитой на три периода. По схожей системе авторы описывают фазы и характер распределения квот в Евросоюзе и периодизацию ограничений, налагаемых на использование кредитов Механизмов Чистого развития и совместного осуществления. Наконец, авторы описывают отношения России и углеродного рынка. Отмечается, что для России важно соблюсти баланс между собственными интересами и интеграцией в международную деятельность по торговле выбросами. Более того, авторы отмечают необходимость методичной работы по развитию углеродных механизмов и собственной инфраструктуры торговли углеродными единицами на примере США и Китая, то есть ввести СТВ в отдельно взятых регионах страны. Хочу отметить, что данная работа является хорошо структурированной, в статье присутствуют комплексно проработанные таблицы, что сильно помогает в понимании описанной проблемы и позволяет выделить особенности торговли углеродными единицами на финансовом рынке. Тем не менее, как было упомянуто выше, в данной работе не рассматриваются отрицательные стороны инвентаризации выбросов, что является допущением.
\subsection{«Углеродная валюта. Новый товар на новых рынках» Ануфриев В. П., Чазов А. В.}
В данной статье авторы рассматривают подписание Киотского протокола, как новую веху в истории международных соглашений в области защиты окружающей среды, которая ознаменовала появление на рынке нового товара, а именно — квоты на выбросы парниковых газов. В материале детально описываются особенности Киотского протокола, а также приводятся статистические доказательства его эффективности, влияния и значимости для развития экономических и экологических отношений стран-участниц соглашения. Авторы подробно разбирают концепцию квот на выбросы как экономический ресурс, описывая место и роль отдельных стран, например, России, Великобритании, Франции, Швейцарии и др. в данной системе. Отдельные подглавы авторы посвящают экологии и инновациям в данной сфере и влиянию на отдельные регионы. Данная статья является полезной для изучения, так как предлагает обширную и подробно разобранную информацию. Важным плюсом является обилие статистических данных. Однако стоит отметить, что у работы есть один важный недостаток — она написана в 2006 году, поэтому предоставляет устаревшую информацию и не дает понимания современной ситуации по рассматриваемому вопросу.
\subsection{«Углеродный рынок Китая глазами зарубежных экспертов» Лукашенко И. В.}
Данная статья посвящена организации углеродного рынка Китая, который является одним из самых эффективно функционирующих на данный момент. Автор анализирует долю Китая в мировом количестве выбросов по годам, а так же пошагово и структурированно описывает последовательность действий китайского правительства для достижения углеродной нейтральности. Например, введение механизма чистого развития, открытие эколого-энергетических бирж, запуск пилотного проекта в некоторых провинциях и другие. Также автор представляет результаты работы пилотной модели по регионам.  Эффективный углеродный рынок решает не только экологическую проблему, но и увеличивает приток инвестиций, укрепляет торговый баланс Китая, как пишет автор статьи. Данная статья является крайне полезной для написания исследования, так как она наилучшим образом описывает работу углеродного рынка, в сравнении с другими предложенными мною статьями.
\subsection{«Углеродное ценообразование как инструмент трансграничного углеродного регулирования и «зеленой» трансформации мировой экономики»  М.В. Лысунец.}
В данной статье М.В. Лысунец рассматривает углеродное ценообразование как инструмент трансграничного углеродного регулирования и «зеленой» трансформации мировой экономики. Автор отмечает, что борьба с изменениями климата остается одной из первостепенных целей социальной, экономической и экологической повестки текущего столетия, а также является одной из семнадцати целей устойчивого развития Организации Объединенных Наций, а также приводит статистику выбросов парниковых газов в мире по секторам экономики и по странам. Далее автор пишет, что в теории углеродного ценообразования существует несколько способов установлении платы за выбросы парниковых газов: Система торговли эмиссионными квотами (Cap-and-trade), Базисно-кредитный подход, Углеродные налоги. Хочу отметить, что автор достаточно подробно описывает международная кооперацию в сфере углеродного регулирования, более того, посвящает углеродное регулирование в России. Среди преимуществ данной работы хочу выделить наличие подробных диаграмм и статистических показателей, а также историческую ретроспективу, которая позволяет проследить этапы борьбы человечества с изменениями климата.
\section{Список литературы}
\begin{itemize}
    \item Хакимова Ж. А., Васильева А. А. К вопросу о правовом регулировании рынка углеродных единиц: сравнительно-правовой аспект //Право и практика. – 2023. – №. 2. – С. 127-131.
    \item Лукашенко И. В. Риски использования углеродной единицы как инструмента финансового рынка //Экономика. Налоги. Право. – 2013. – №. 4. – С. 50-55.
    \item Александрова Ж. П., Кат С. А. Достижение углеродной нейтральности путем внедрения инструментов налогообложения //Кронос. – 2022. – Т. 7. – №. 10 (72). – С. 92-95.
    \item Головач О. В. УГЛЕРОДНЫЕ ЕДИНИЦЫ КАК ОБЪЕКТ БУХГАЛТЕРСКОГО УЧЕТА: ПРИЗНАНИЕ, ОЦЕНКА, МЕТОДИКИ //Известия Санкт-Петербургского государственного экономического университета. – 2023. – №. 4 (142). – С. 73-80.
    \item Тайлаков В. О. Моделирование динамики цен единиц сокращенных выбросов на углеродном рынке //Вестник Кузбасского государственного технического университета. – 2007. – №. 1. – С. 112-113.
    \item Иванов Н. А. Мировой углеродный рынок в стадии зарождения //URL: https://preprints. ru/files/612 (дата обращения 27.06. 2021).
    \item Небесная А. Ю. КОНЦЕПЦИЯ УГЛЕРОДНОГО РЕГУЛИРОВАНИЯ И БИРЖЕВОГО РЫНКА УГЛЕРОДНЫХ ЕДИНИЦ В РЕГИОНАХ РОССИИ //Трансформация экономических систем: низкоуглеродная экономика и климатическая политика. – 2022. – С. 62-67.
    \item Скворцова М. А., Тяглов С. Г. Формирование и развитие российского рынка углеродных единиц //Journal of Economic Regulation (Вопросы регулирования экономики). – 2022. – Т. 13. – №. 4. – С. 89-98.
    \item Рубцов Б. Б., Лукашенко И. В. Особенности торговли углеродными единицами на финансовом рынке //Финансы, деньги, инвестиции. – 2013. – №. 3. – С. 18-26.
    \item Ануфриев В. П., Чазов А. В. Углеродная валюта. Новый товар на новых рынках //Всероссийский экономический журнал ЭКО. – 2006. – №. 1 (379). – С. 61-77.
    \item Лукашенко И. В., Сайфетдинова А. Ф. Углеродный рынок Китая глазами зарубежных экспертов //Финансы: теория и практика. – 2013. – №. 6. – С. 112-121.
    \item Лысунец М. В. Углеродное ценообразование как инструмент трансграничного углеродного регулирования и «зеленой» трансформации мировой экономики //Мир новой экономики. – 2023. – Т. 17. – №. 2. – С. 27-36.
\end{itemize}
\end{document}